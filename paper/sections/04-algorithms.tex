\section{Algorithm Design}
\label{sec:algorithms}

We present four CBS variants that address different aspects of the MD-MAPF problem: \mixedcbs{} for dimensional conflicts, \ecbs{} for energy constraints, \deadline{} for deadline awareness, and \hybridcbs{} for integrated planning-execution.

\subsection{\mixedcbs{}: Dimensional Conflict Resolution}

\mixedcbs{} extends CBS with dimension-aware conflict classification and specialized resolution strategies.

\begin{algorithm}
\caption{\mixedcbs{} Algorithm}
\label{alg:mixedcbs}
\begin{algorithmic}[1]
\REQUIRE MD-MAPF instance $\mathcal{I}$
\ENSURE Solution $\pi$ or \textsc{Failure}
\STATE $\text{root} \leftarrow \text{compute\_initial\_assignment}(\mathcal{I})$
\STATE $\text{root.paths} \leftarrow \text{plan\_all\_paths}(\text{root}, \mathcal{I})$
\STATE $\text{OPEN} \leftarrow \{\text{root}\}$
\WHILE{$\text{OPEN} \neq \emptyset$}
    \STATE $N \leftarrow \text{extract\_min}(\text{OPEN})$
    \STATE $\text{conflict} \leftarrow \text{find\_first\_conflict}(N.\text{paths})$
    \IF{$\text{conflict} = \emptyset$}
        \RETURN $N$ \COMMENT{Solution found}
    \ENDIF
    \STATE $\text{dim} \leftarrow \text{classify\_dimension}(\text{conflict})$
    \STATE $\text{children} \leftarrow \text{resolve\_by\_dimension}(N, \text{conflict}, \text{dim})$
    \FOR{each child in children}
        \STATE $\text{child.paths} \leftarrow \text{replan\_affected}(\text{child}, \text{conflict.agents})$
        \IF{$\text{child.paths} \neq \emptyset$}
            \STATE $\text{OPEN} \leftarrow \text{OPEN} \cup \{\text{child}\}$
        \ENDIF
    \ENDFOR
\ENDWHILE
\RETURN \textsc{Failure}
\end{algorithmic}
\end{algorithm}

The key innovation is the \texttt{resolve\_by\_dimension} function, which applies conflict-class-specific strategies:

\textbf{Strategy $C_1$ (\textsc{Linear}):} For rail robot conflicts on segment $S$:
\begin{equation}
\text{resolve\_linear}(a_1, a_2, S, t) = \text{constrain}(a_{\text{trailing}}, \text{wait\_at\_junction}, t)
\end{equation}
Rail robots cannot pass on a segment; one must yield at a junction.

\textbf{Strategy $C_2$ (\textsc{Planar}):} Standard CBS branching:
\begin{equation}
\text{resolve\_planar}(a_1, a_2, v, t) = \{\text{constrain}(a_1, v, t), \text{constrain}(a_2, v, t)\}
\end{equation}

\textbf{Strategy $C_3$ (\textsc{Crossing}):} Rail-mobile intersection with priority to rail (higher inertia):
\begin{equation}
\text{resolve\_crossing}(a_{\text{rail}}, a_{\text{mobile}}, v, t) = \{\text{constrain}(a_{\text{mobile}}, v, [t-\epsilon, t+\epsilon])\}
\end{equation}

\textbf{Strategy $C_4$ (\textsc{Aerial}):} Similar to planar but in 3D layer context.

\textbf{Strategy $C_5$ (\textsc{Vertical}):} Corridor exclusivity forces sequentialization:
\begin{equation}
\text{resolve\_vertical}(a_1, a_2, C, t) = \text{constrain}(a_{\text{following}}, C, [t, t + \tau_{\text{traverse}}])
\end{equation}

\textbf{Strategy $C_6$ (\textsc{Air-Ground}):} Synchronization at handoff points with landing coordination.

\subsection{\ecbs{}: Energy-Constrained CBS}

\ecbs{} extends \mixedcbs{} with energy feasibility checking and automatic charging station waypoint insertion.

\begin{algorithm}
\caption{\ecbs{} Algorithm}
\label{alg:ecbs}
\begin{algorithmic}[1]
\REQUIRE MD-MAPF instance $\mathcal{I}$ with energy parameters
\ENSURE Energy-valid solution $\pi$ or \textsc{Failure}
\STATE Initialize as \mixedcbs{}
\WHILE{$\text{OPEN} \neq \emptyset$}
    \STATE $N \leftarrow \text{extract\_min}(\text{OPEN})$
    \STATE \textbf{// Energy validation before conflict detection}
    \STATE $\epsilon \leftarrow \text{simulate\_energy}(N.\text{paths}, \mathcal{I})$
    \IF{$\epsilon \neq \emptyset$}
        \STATE $\text{children} \leftarrow \text{resolve\_energy\_violation}(N, \epsilon)$
        \STATE Add valid children to OPEN
        \STATE \textbf{continue}
    \ENDIF
    \STATE \textbf{// Standard conflict resolution}
    \STATE $\text{conflict} \leftarrow \text{find\_first\_conflict}(N.\text{paths})$
    \IF{$\text{conflict} = \emptyset$}
        \RETURN $N$
    \ENDIF
    \STATE Resolve conflict as in \mixedcbs{}
\ENDWHILE
\RETURN \textsc{Failure}
\end{algorithmic}
\end{algorithm}

\begin{definition}[Energy Violation]
An energy violation is a triple $\epsilon = (a, t_{\text{depleted}}, v_{\text{depleted}})$ indicating agent $a$ will exhaust energy at time $t_{\text{depleted}}$ at position $v_{\text{depleted}}$.
\end{definition}

The resolution inserts a charging waypoint:
\begin{align}
\text{pad} &\leftarrow \text{find\_nearest\_charging\_pad}(v_{\text{depleted}}) \\
t_{\text{reach}} &\leftarrow t_{\text{depleted}} - \text{safety\_margin} \\
\text{goals}[a] &\leftarrow \text{insert\_waypoint}(\text{pad}, \text{before\_tasks})
\end{align}

\subsection{\deadline{}: Deadline-Aware CBS}

\deadline{} introduces slack-based prioritization to ensure deadline feasibility.

\begin{definition}[Slack-Weighted Cost]
The CBS node cost function is modified:
\begin{equation}
\text{cost}(N) = \text{makespan}(N) + \alpha \sum_{t \in T} \max(0, -\text{slack}(t))
\end{equation}
where $\alpha$ penalizes deadline violations.
\end{definition}

The algorithm preferentially expands nodes where all tasks meet deadlines, using slack as a tiebreaker. When conflicts arise, the agent with less slack receives priority in constraint assignment.

\subsection{\hybridcbs{}: Planning-Execution Integration}

\hybridcbs{} combines global CBS planning with local potential field execution, enabling real-time adaptation while maintaining optimality guarantees.

\begin{definition}[Potential Field]
A potential field is a function $\Phi: V \rightarrow \mathbb{R}$:
\begin{equation}
\Phi(v) = \Phi_{\text{goal}}(v) - \Phi_{\text{repulsive}}(v) + \Phi_{\text{thermal}}(v)
\end{equation}
where:
\begin{itemize}
    \item $\Phi_{\text{goal}}(v)$: Attraction toward goal/task locations
    \item $\Phi_{\text{repulsive}}(v)$: Repulsion from other agents
    \item $\Phi_{\text{thermal}}(v)$: Repulsion from overheated zones
\end{itemize}
\end{definition}

\begin{definition}[Modified A* Heuristic]
The low-level A* heuristic incorporates the field:
\begin{equation}
h'(v) = h(v) - \lambda \cdot \Phi(v)
\end{equation}
where $\lambda \in [0, 1]$ controls field influence.
\end{definition}

\begin{definition}[Deviation Detection]
Deviation of agent $a$ at time $t$ is:
\begin{equation}
\text{dev}(a, t) = \text{dist}(\text{actual\_pos}(a, t), \text{planned\_pos}(a, t))
\end{equation}
\end{definition}

When $\text{dev}(a, t) > \theta_{\text{replan}}$, local replanning is triggered:
\begin{equation}
\text{local\_path} \leftarrow \text{field\_guided\_astar}(\text{actual\_pos}, \text{planned\_pos}(t + H))
\end{equation}
where $H$ is the planning horizon.

\subsection{Slack-Field Mapping}

The critical bridge between planning and execution maps deadline slack to the \ekkor{} field coordination mechanism:

\begin{equation}
\text{FIELD\_SLACK} = \text{clamp}\left(\frac{\text{slack}(t)}{\tau_{\text{normalize}}}, 0, 1\right)
\end{equation}

where $\tau_{\text{normalize}} = 100$ seconds provides normalization.

The gradient of the slack field across neighbors indicates deadline pressure:
\begin{equation}
\nabla\text{FIELD\_SLACK} = \frac{1}{k}\sum_{j \in \mathcal{N}_k} (\text{slack}_j - \text{slack}_i)
\end{equation}

Negative gradient indicates neighbors have tighter deadlines, signaling the local module to offer assistance or yield resources.
