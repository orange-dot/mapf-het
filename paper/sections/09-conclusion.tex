\section{Conclusion}
\label{sec:conclusion}

This paper presented \mapfhet{}, a unified framework for heterogeneous multi-agent coordination spanning from optimal planning to distributed real-time execution.

\subsection{Summary of Contributions}

\textbf{MD-MAPF Formulation:} We introduced the first MAPF formulation with explicit dimensional modeling. The dimensionality function $\kappa(a) \rightarrow \{1,2,3\}$ and vertex compatibility function $\delta(v)$ enable precise representation of agents operating in 1D rail, 2D floor, and 3D airspace environments within a unified planning framework.

\textbf{Dimensional Conflict Taxonomy:} The six-class taxonomy (LINEAR, PLANAR, CROSSING, AERIAL, VERTICAL, AIR-GROUND) provides specialized resolution strategies that exploit dimensional structure. Experimental results show 5.6$\times$ speedup over baseline CBS while maintaining optimal makespan.

\textbf{Planning-Execution Bridge:} The slack field mechanism maps planning deadline constraints to execution layer gradients, enabling distributed coordination that respects global optimality while adapting to local conditions. Zero energy violations and 0.1\% deadline violation rate demonstrate effective integration.

\textbf{Scale-Free Coordination:} The $k=7$ topological neighbor approach, grounded in collective behavior research, achieves $O(\log N)$ convergence scaling. At 2048 modules, load balancing completes in 520ms with 42\% bandwidth utilization—a 9$\times$ reduction compared to full-mesh alternatives.

\subsection{Limitations}

The current implementation assumes static dimensional assignments ($\kappa$ is constant). Reconfigurable systems where agents can change dimensionality (e.g., a ground robot that deploys a drone) would require extensions to the MD-MAPF formulation.

The $k=7$ neighbor constant, while empirically validated, may not be optimal for all network topologies. Adaptive neighbor selection based on network characteristics remains future work.

Energy modeling uses discrete action categories. Continuous energy models accounting for payload, wind, and battery degradation would improve accuracy for long-horizon planning.

\subsection{Future Work}

\textbf{Dynamic Dimensionality:} Extending MD-MAPF to handle agents that can transition between dimensional modes (e.g., amphibious vehicles, deployable drones) would broaden applicability.

\textbf{Learning-Based Coordination:} Integrating learned policies for field weight adaptation and neighbor selection could improve performance in specific deployment scenarios while maintaining safety guarantees through formal verification of learned components.

\textbf{Standardization:} The planning-execution interface (slack field mapping, capability bitmasks) could form the basis for an open standard enabling interoperability between planning systems and embedded controllers from different vendors.

\textbf{Hardware Validation:} While our embedded implementations target STM32G474 and AURIX TC375, deployment on alternative platforms (RISC-V, custom ASICs) would validate portability claims and enable broader adoption.

\subsection{Broader Impact}

The techniques presented apply beyond electric vehicle charging to any domain requiring coordination of heterogeneous agents with deadline constraints: warehouse automation, port logistics, construction robotics, and agricultural fleets. The open-source implementation (planning in Go, execution in C/Rust) provides a foundation for both research extensions and commercial deployment.

By unifying planning optimality with execution scalability, \mapfhet{} addresses a fundamental gap in multi-robot systems, enabling reliable coordination of heterogeneous fleets at scales previously achievable only through centralized control.
